% ОБЯЗАТЕЛЬНО ИМЕННО ТАКОЙ documentclass!
% (Основной кегль = 14pt, поэтому необходим extsizes)
% Формат, разумеется, А4
% article потому что стандарт не подразумевает разделов
% Глава = section, Параграф = subsection
% (понятия "глава" и "параграф" из документа, описывающего диплом)
\documentclass[a4paper,14pt]{extarticle}

% Подключаем главный пакет со всем необходимым
\usepackage{diploma}

% Пакеты по желанию (самые распространенные)
% Хитрые мат. символы
\usepackage{euscript}
% Таблицы
\usepackage{longtable}
\usepackage{makecell}
% Картинки (можно встявлять даже pdf)
\usepackage[pdftex]{graphicx}

\usepackage{amsthm,amssymb, amsmath}
\usepackage{textcomp}

% Подсветка кода (все стили в файле)
\input{code_highlight.tex}

\begin{document}

% Титульник в файле titlepage.tex
\newgeometry{left=30mm, top=20mm, right=15mm, bottom=20mm, nohead, nofoot}
\begin{titlepage}
\begin{center}
Министерство высшего образования и науки Российской федерации

Федеральное государственное автономное \\образовательное учреждение высшего образования

\textbf{<<Национальный исследовательский ядерный университет}
\textbf{<<МИФИ>>}

\vspace{25mm}

\textbf{\textit{\large Фамилия Имя Отчество}} \\[8mm]
% Название
\textbf{\large Выпускная квалификационная работа}\\[3mm]
\textbf{\textit{\large Название работы}}

\vspace{10mm}
Уровень образования: бакалавриат / магистратура\\
Направление 11.04.04 «Электроника и наноэлектроника»\\
Образовательная программа
«Наноэлектроника, спинтроника и фотоника»

\vspace{15mm}

% Научный руководитель, рецензент
\begin{flushleft}
\textbf{Выпускник:} Фамилия И.О.

\hspace{10cm} \textit{Подпись}: \space \hrulefill

\textbf{Научный руководитель:} 

к.ф.-м.н., доцент кафедры физики конденсированных сред

ИНТЭЛ НИЯУ МИФИ, Фамилия И.О.

\hspace{10cm} \textit{Подпись}: \space \hrulefill

\textbf{И.о. заместителя заведующего кафедрой:} 

д.ф.-м.н., профессор кафедры физики конденсированных сред 

ИНТЭЛ НИЯУ МИФИ, Никитенко В.Р.

\hspace{10cm} \textit{Подпись}: \space \hrulefill \space
\end{flushleft}

\vfill 

{Москва}
\par{\the\year{} г.}
\end{center}
\end{titlepage}
% Возвращаем настройки geometry обратно (то, что объявлено в преамбуле)
\restoregeometry
% Добавляем 1 к счетчику страниц ПОСЛЕ titlepage, чтобы исключить 
% влияние titlepage environment
\addtocounter{page}{1}


% Содержание
\tableofcontents
\pagebreak

% ============================================
% ВВЕДЕНИЕ
% ============================================
\specialsection{Введение}

Здесь необходимо рассказать, о чём работа. Объём 1-2 страницы. Нужно охарактеризовать область исследования, практическую значимость (для разработки каких приборов могут быть использованы ваши результаты), какую проблему решает ваша работа (кратко, подробнее будет в обзоре), какие методы использованы (тоже кратко, подробнее в главе Методы).

Здесь необходимо рассказать, о чём работа. Объём 1-2 страницы. Нужно охарактеризовать область исследования, практическую значимость (для разработки каких приборов могут быть использованы ваши результаты), какую проблему решает ваша работа (кратко, подробнее будет в обзоре), какие методы использованы (тоже кратко, подробнее в главе Методы).

Здесь необходимо рассказать, о чём работа. Объём 1-2 страницы. Нужно охарактеризовать область исследования, практическую значимость (для разработки каких приборов могут быть использованы ваши результаты), какую проблему решает ваша работа (кратко, подробнее будет в обзоре), какие методы использованы (тоже кратко, подробнее в главе Методы).

\specialsection{Цель и задачи}
\label{Tasks}

\textbf{Цель:} Цель не должна совпадать с темой работы. Цель должна быть достижима (должен быть конечный результат) и проверяема. Исследование --- это процесс, и целью быть не может.

\textbf{Задачи}
\begin{enumerate}
    \item Задача 1
    \item Задача 2
    \item Задача 3
    \item Задача 4
\end{enumerate}

Достаточно задач. Обзор литературы наверное в задачи включать не будем. Лучше написать конкретно, что мы делаем (разработка алгоритма, программная реализация, расчёт конкретных параметров при определённых условиях и т.д.)

% ============================================
% ГЛАВА 1
% ============================================
\pagebreak
\section{Обзор литературы}
\label{Review}

Ненумерованная формула:

\begin{equation}
    \begin{pmatrix} \dot{\varphi}\\ \dot{\theta} \\ \dot{\psi} \end{pmatrix}
    = \begin{pmatrix}
        cos(\theta)cos(\psi) & -sin(\psi) & 0 \\
        cos(\theta)sin(\psi) & cos(\psi)  & 0 \\
        -sin(\theta)         & 0         &  1
    \end{pmatrix}^{-1}
    \begin{pmatrix} \omega_x\\ \omega_y \\ \omega_z \end{pmatrix}.
\end{equation}

Нумерованная формула:

\begin{equation}
    i^2 = -1.
    \label{eq:my_ref}
\end{equation}

Тест ссылки на формулу \ref{eq:my_ref}.

\begin{lstlisting}[language=rust,caption={Программная реализация метода Рунге-Кутты},label={listing-1}]
// From the pendulum program
fn runge_kutta(
    vars: &MyVec,
    pars: &Vec<f64>,
    rhs: &dyn Fn(&MyVec, &Vec<f64>) -> MyVec,
    dt: f64,
) -> MyVec {
    let rk_1 = rhs(vars, pars);
    let rk_2 = rhs(&vars.add(&rk_1.scale(dt / 2.0)), pars);
    let rk_3 = rhs(&vars.add(&rk_2.scale(dt / 2.0)), pars);
    let rk_4 = rhs(&vars.add(&rk_3.scale(dt)), pars);

    let vars_new = vars
        .add(&rk_1.scale(dt / 6.0))
        .add(&rk_2.scale(dt / 3.0))
        .add(&rk_3.scale(dt / 3.0))
        .add(&rk_4.scale(dt / 6.0));
    vars_new
}
\end{lstlisting}

\begin{lstlisting}[language=C++,caption={Подпрограмма случайного блуждания на плоскости},label={listing-2}]
std::random_device rd;
std::mt19937 mt(rd());
std::uniform_int_distribution<long> dist(1, 4);
std::vector<long> xn(n0, 0);
std::vector<long> yn(n0, 0);
for (long jt = 0; jt < M; jt++)
{
    for (long jn = 0; jn < n0; jn++)
    {
        switch (dist(mt))
        {
        case 1:
            xn[jn] ++;
            break;
        case 2:
            xn[jn] --;
            break;
        case 3:
            yn[jn] ++;
            break;
        case 4:
            yn[jn] --;
            break;
        }
    }
}
\end{lstlisting}

Ниже тестируется очень большая таблица на несколько страниц

\begin{center}
    \begin{longtable}{|p{2cm}|p{3cm}|p{7cm}|p{3cm}|}
    \caption{Заголовок таблицы}\\
    \hline
    1 & 2 & 3 & 4\\ 
    \hline 
    2 & 2 & 3 & 4\\
    \hline
    3 & 2 & 3 & 4\\
    \hline
    4 & 2 & 3 & 4\\
    \hline
    5 & 2 & 3 & 4\\
    \hline
    6 & 2 & 3 & 4\\
    \hline
    7 & 2 & 3 & 4\\
    \hline
    8 & 2 & 3 & 4\\
    \hline
    9 & 2 & 3 & 4\\
    \hline
    10 & 2 & 3 & 4\\
    \hline
    
    
    \end{longtable}
\end{center}


А также тестируется счетчик таблиц, жирные и двойные линии.

\begin{center}
    \begin{longtable}{|p{2cm}||p{3cm}|p{7cm}|p{3cm}|}
    \caption{Заголовок таблицы нумер 2}\\
    \hline
    1 & 2 & 3 & 4\\ 
    \hline
    2 & 2 & 3 & 4\\
    \hline
    3 & 2 & очень жирная ячейка \par с переносом & 4\\
    \hline
    4 & 2 & 3 & 4\\
    \hline
    5 & 2 & 3 & 4\\
    \hline
    6 & 2 & 3 & 4\\
    \hline
    7 & 2 & 3 & 4\\
    \hline
    8 & 2 & 3 & 4\\
    \hline
    9 & 2 & 3 & 4\\
    \hline
    10 & 2 & 3 & 4\\
    \hline
    
    
    \end{longtable}
\end{center}

Ссылаемся на Листинг \ref{listing-1} здесь.
% ============================================
% ГЛАВА 2
% ============================================
\pagebreak
\section{Теория и основные уравнения}

% ============================================
% ГЛАВА 3
% ============================================
\pagebreak
\section{Численные методы и алгоритмы}

% ============================================
% ГЛАВА 4
% ============================================
\pagebreak
\section{Программная реализация}

% ============================================
% ГЛАВА 5
% ============================================
\pagebreak
\section{Результаты и обсуждение}

\begin{figure}[ht]
\begin{center}
\scalebox{0.4}{
   \includegraphics{images/graph.jpg}
}

\caption{
\label{graph-fig}
     Линейные функции.}
\end {center}
\end {figure}

Ссылаемся на график ~\ref{graph-fig}.
Ссылка на статью: \cite{voc}, \cite{vo2}

\specialsection{Выводы}
Жизнь --- тлен.
\pagebreak

\specialsection{Заключение}



% Библиография в cpsconf стиле
% Аргумент {1} ниже включает переопределенный стиль с выравниванием слева
\begin{thebibliography}{1}
\bibitem{voc} Griffin D.W., Lim J.S. \flqq Multiband excitation vocoder\frqq. IEEE ASSP-36 (8), 1988, pp. 1223-1235.
\bibitem{vo2} Griffin D.W., Lim J.S. \flqq Multiband excitation vocoder\frqq. IEEE ASSP-36 (8), 1988, pp. 1223-1235.
\end{thebibliography}
\end{document}